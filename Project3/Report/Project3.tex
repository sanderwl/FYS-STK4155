\documentclass[12pt,a4paper]{article}
\usepackage[utf8]{inputenc}

\usepackage{amsmath}
\usepackage{amsfonts}
\usepackage{amssymb}
\usepackage{graphicx}
\usepackage{listings}
\usepackage[margin=1.0in]{geometry}
\usepackage{caption}
\usepackage{subcaption}
\usepackage{float}
\usepackage[utf8]{inputenc}
\usepackage{refstyle}
\usepackage{spverbatim}
\usepackage{listings}
\usepackage{csvsimple}
\usepackage{adjustbox}
\usepackage{cancel}
\usepackage{scalerel,stackengine}
\stackMath
\newcommand\reallywidehat[1]{%
\savestack{\tmpbox}{\stretchto{%
  \scaleto{%
    \scalerel*[\widthof{\ensuremath{#1}}]{\kern-.6pt\bigwedge\kern-.6pt}%
    {\rule[-\textheight/2]{1ex}{\textheight}}%WIDTH-LIMITED BIG WEDGE
  }{\textheight}% 
}{0.5ex}}%
\stackon[1pt]{#1}{\tmpbox}%
}
\parskip 1ex

\lstset{numbers=left,
	title=\lstname,
	numberstyle=\tiny, 
	breaklines=true,
	tabsize=4,
	language=Python,
	morekeywords={with,super,as},,
	frame=single,
	basicstyle=\footnotesize\tt,
	commentstyle=\color{comment},
	keywordstyle=\color{keyword},
	stringstyle=\color{string},
	backgroundcolor=\color{background},
	showstringspaces=false,
	numbers=left,
	numbersep=5pt,
	literate=
		{æ}{{\ae}}1
		{å}{{\aa}}1
		{ø}{{\o}}1
		{Æ}{{\AE}}1
		{Å}{{\AA}}1
		{Ø}{{\O}}1
	}

\usepackage{bm}
\usepackage{hyperref}
\usepackage[usenames, dvipsnames]{color}

\begin{document}

\begin{center}
\LARGE{\textbf{Project 2: Classification and Regression, from linear and logistic regression to neural networks}}
\\
\large{\textbf{Course: FYS-STK4155}}
\\
\large{\textbf{Semester: Autumn 2020}}
\\
\large{\textbf{Name: Sander Losnedahl}}
\end{center}

\begin{center}
\Large{\textbf{Abstract}}
\end{center}

\noindent a

\newpage

\begin{center}
\Large{\textbf{Introduction}}
\end{center}

\noindent a

\newpage

\begin{center}
\Large{\textbf{Preliminaries}}
\end{center}

\noindent a

\newpage

\begin{center}
\Large{\textbf{Exercise 1a): a}}
\end{center}

\begin{center}
\large{\textbf{The heat equation}}
\end{center}

\noindent In this exercise we want to mathematically describe how heat can transfer within a rod of length L over a given time interval t. The rod will initially be heated at time $t = 0$ and the heat will decay after this time. How the heat changes as function of time and space is given by the heat equation as seen below.

\begin{equation}\label{eq:heatEquation}
\frac{\partial u(x,t)}{\partial t} = \frac{K_0}{c\rho} \frac{\partial^2 u(x,t)}{\partial x^2} 
\end{equation}

\noindent where $u(x,t)$ is the temperature at a specific time t and a position x. The term $\frac{K_0}{c\rho}$ is the thermal diffusivity which consists of thermal conductivity $K_0$ divided by the specific heat capacity c times the density of the material $\rho$. This quantity is a measure of the rate of heat transfer from the heated part of the rod to the cold part of the rod. 
\\
Equation \ref{eq:heatEquation} is a partial differential equation (PDF) which means that we need an initial condition and two boundary conditions in order to solve the equation. The initial condition

\newpage

\begin{center}
\Large{\textbf{Exercise 1b: b}}
\end{center}

\begin{center}
\large{\textbf{a}}
\end{center}

\noindent a

\newpage

\begin{center}
\Large{\textbf{Exercise d): a}}
\end{center}

\begin{center}
\large{\textbf{a}}
\end{center}

\noindent a

\newpage

\begin{center}
\Large{\textbf{Exercise e): a}}
\end{center}

\noindent a

\newpage

\begin{center}
\Large{\textbf{Conclusion}}
\end{center}

\noindent a

\newpage

\begin{center}
\Large{\textbf{Future work}}
\end{center}

\noindent a

\newpage

\begin{center}
\Large{\textbf{References}}
\end{center}

\begin{itemize}
  \item a
\end{itemize}

\end{document}
